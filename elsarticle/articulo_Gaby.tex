%% 
%% Copyright 2007-2019 Elsevier Ltd
%% 
%% This file is part of the 'Elsarticle Bundle'.
%% ---------------------------------------------
%% 
%% It may be distributed under the conditions of the LaTeX Project Public
%% License, either version 1.2 of this license or (at your option) any
%% later version.  The latest version of this license is in
%%    http://www.latex-project.org/lppl.txt
%% and version 1.2 or later is part of all distributions of LaTeX
%% version 1999/12/01 or later.
%% 
%% The list of all files belonging to the 'Elsarticle Bundle' is
%% given in the file `manifest.txt'.
%% 
%% Template article for Elsevier's document class `elsarticle'
%% with harvard style bibliographic references

%\documentclass[preprint,12pt,twocolumns]{elsarticle}

%% Use the option review to obtain double line spacing
%% \documentclass[preprint,review,12pt]{elsarticle}

%% Use the options 1p,twocolumn; 3p; 3p,twocolumn; 5p; or 5p,twocolumn
%% for a journal layout:
%% \documentclass[final,1p,times]{elsarticle}
%% \documentclass[final,1p,times,twocolumn]{elsarticle}
%% \documentclass[final,3p,times]{elsarticle}
%% \documentclass[final,3p,times,twocolumn]{elsarticle}
 \documentclass[final,5p,times]{elsarticle}
%% \documentclass[final,5p,times,twocolumn]{elsarticle}

%% For including figures, graphicx.sty has been loaded in
%% elsarticle.cls. If you prefer to use the old commands
%% please give \usepackage{epsfig}

%% The amssymb package provides various useful mathematical symbols
\usepackage{amssymb}
\usepackage[spanish]{babel}
%% The amsthm package provides extended theorem environments
%% \usepackage{amsthm}

%% The lineno packages adds line numbers. Start line numbering with
%% \begin{linenumbers}, end it with \end{linenumbers}. Or switch it on
%% for the whole article with \linenumbers.
%% \usepackage{lineno}

%\journal{Nuclear Physics B}

\begin{document}

\begin{frontmatter}

%% Title, authors and addresses

%% use the tnoteref command within \title for footnotes;
%% use the tnotetext command for theassociated footnote;
%% use the fnref command within \author or \address for footnotes;
%% use the fntext command for theassociated footnote;
%% use the corref command within \author for corresponding author footnotes;
%% use the cortext command for theassociated footnote;
%% use the ead command for the email address,
%% and the form \ead[url] for the home page:
%% \title{Title\tnoteref{label1}}
%% \tnotetext[label1]{}
%% \author{Name\corref{cor1}\fnref{label2}}
%% \ead{email address}
%% \ead[url]{home page}
%% \fntext[label2]{}
%% \cortext[cor1]{}
%% \address{Address\fnref{label3}}
%% \fntext[label3]{}

\title{Violencia intrafamiliar y comunitaria en Nuevo Le\'on}

%% use optional labels to link authors explicitly to addresses:
%% \author[label1,label2]{}
%% \address[label1]{}
%% \address[label2]{}

\author{Gabriela S\'anchez Yepez}

\address{Posgrado en Ingenier\'ia de Sistemas \\
Facultad de Ingenier\'ia Mec\'anica y El\'ectrica\\
Universidad Aut\'onoma de Nuevo Le\'on}

\begin{abstract}
Se presenta un an\'alisis de incidentes de violencia intrafamiliar y comunitaria en el estado de Nuevo Le\'on reportados durante el a\~no 2018. Es importante identificar los grupos vulnerables y posibles causas al problema que permitan planear estrategias de prevenci\'on. Para realizar dicho an\'alisis, se utilizan distintas herramientas que proporciona el lenguaje de programaci\'on Python. %Se busca identificar el grupo m\'as vulnerable a presentar violencia familiar as\'i c\'omo las posibles causas co el objetivo de  %Como principal objetivo se busca identificar el grupo m\'as vulnerable para planear estrategias de prevenci\'on.
\end{abstract}

%%Graphical abstract
%\begin{graphicalabstract}
%\includegraphics{grabs}
%\end{graphicalabstract}

%%Research highlights
%\begin{highlights}
%\item Research highlight 1
%\item Research highlight 2
%\end{highlights}

\begin{keyword}
%% keywords here, in the form: keyword \sep keyword

%% PACS codes here, in the form: \PACS code \sep code

%% MSC codes here, in the form: \MSC code \sep code
%% or \MSC[2008] code \sep code (2000 is the default)

\end{keyword}

\end{frontmatter}

%% \linenumbers

%% main text
%\twocolumn[
%\begin{@twocolumnfalse}
\section{Introducci\'on} \label{intro}

Este trabajo busca aplicar herramientas de ciencia de datos que permitan analizar el efecto de distintos factores presentes en incidentes de violencia intrafamiliar y comunitaria en el estado de Nuevo Le\'on. 

La estructura del art\'iculo es la siguiente: * 
\section{Antecedentes} \label{antecedentes}

\section{Literatura relacionada} \label{estado_arte}

\section{Metodolog\'ia} \label{metodologia}

En esta secci\'on se especifican las car\'acteristicas los datos con los que se trabaja as\'i como las herramientas utilizadas para el an\'alisis de dichos datos. 

\subsection*{Datos}

Los datos fueron proporcionados por la doctora Patricia L. Cerda, en formato xlsx.

Los datos se presentan en dos archivos, el primero contiene los incidentes de violencia intrafamiliar y comunitaria presentados durante los meses de enero a noviembre en el a\~no 2018 y el segundo, los incidentes del resto del a\~no, teniendo un total de 16 410 reportes. 

Los registros contienen 47 columnas de las cuales \'unicamente 16 proporcionan informaci\'on relevante, sin datos personales, que ser\'an usados en el an\'alisis. Estos registros proveen datos sobre lugar, fecha y hora de los incidentes, as\'i como informaci\'on sobre la(s) v\'ictima(s) y agresor(es).  

\subsection*{Herramientas}

Para realizar un pre-procesamiento de los datos primero se utiliz\'o la herramienta \textsc{bash} que permiti\'o convertir el archivo de los datos a un formato manejable, esto es, csv.

La siguiente fase de pre-procesamiento se realiz\'o con la librer\'ia \textsc{Pandas}. Con \'esta, se utiliz\'o para unir los datos en un solo archivo, eliminar la informaci\'on que no se utiliza y cambiar los tipos de los datos. 

\section{Resultados} \label{resultados}

\section{Conclusiones} \label{conclusiones}

%% The Appendices part is started with the command \appendix;
%% appendix sections are then done as normal sections
%% \appendix

%% \section{}
%% \label{}

%% For citations use: 
%%       \citet{<label>} ==> Jones et al. [21]
%%       \citep{<label>} ==> [21]
%%

%% If you have bibdatabase file and want bibtex to generate the
%% bibitems, please use
%%
%%  \bibliographystyle{elsarticle-num-names} 
%%  \bibliography{<your bibdatabase>}

%% else use the following coding to input the bibitems directly in the
%% TeX file.

\begin{thebibliography}{00}

%% \bibitem[Author(year)]{label}
%% Text of bibliographic item

%\bibitem[ ()]{}

\end{thebibliography}

\section*{Agradecimientos}

Agradezco al Consejo Nacional de Ciencia y Tecnolog\'ia (CONACyT) por la beca otorgada. A la doctora Patricia L. Cerda por proveer los datos utilizados en el estudio y a la doctora Elisa Schaeffer por la gu\'ia proporcionada para la realizaci\'on de este trabajo.


%\end{@twocolumnfalse}]
\end{document}

\endinput
%% End of file `elsarticle-template-num-names.tex'.
